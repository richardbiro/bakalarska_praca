\chapter*{Úvod} % chapter* je necislovana kapitola
\addcontentsline{toc}{chapter}{Úvod} % rucne pridanie do obsahu
\markboth{Úvod}{Úvod} % vyriesenie hlaviciek

Magické útvary zaujímali ľudí už odpradávna. \\

Najznámejším z nich je magický štvorec. Ide o štvorcovú tabuľku veľkosti $3 \times 3$ vyplnenú číslami, pričom platí, že súčet čísel v riadku, stĺpci a na oboch diagonálach je konštantný. \\

Medzi ďalšie známe magické útvary patria obdĺžniky, grafy, kruhy alebo hviezdy. \\

Existujú však aj iné magické vlastnosti, ktoré môžeme na útvaroch skúmať. V štvorci nás môže zaujímať súčin prvkov a nie ich súčet (v takom prípade ide o multiplikatívny štvorec). Alebo budeme brať do úvahy súčet prvkov aj ich súčet druhých mocnín (vtedy ide o bimagický štvorec). \\

O niektorých útvaroch s konkrétnymi magickými vlastnosťami stále nevieme povedať, či existujú. Medzi najväčšie otvorené problémy patrí existencia magického štvorca veľkosti $3 \times 3$, ktorého prvky sú navzájom rôzne a sú druhými mocninami kladných celých čísel. Bolo dokázané prepojenie tohto problému s aritmetickými postupnosťami, kongruentnými číslami a eliptickými krivkami. \\

V kapitole \ref{kap:definitions} uvedieme základné pojmy a definície pri práci s útvarmi, ako aj súčasný stav danej problematiky. \\

V kapitole \ref{kap:openproblems} podrobnejšie preskúmame niektoré známe otvorené problémy z oblasti magických útvarov. \\

Magické vlastnosti niektorých útvarov ešte stále neboli preskúmané. V kapitole \ref{kap:newproblems} sa pozrieme na nové definované problémy z oblasti magických útvarov. Zadefinujeme tieto nové typy útvarov: vrcholovo a hranovo bimagické grafy, vrcholovo a hranovo multiplikatívne magické grafy, bimagické a multiplikatívne magické obdĺžniky. Ku každému z nich uvedieme zistenia a nutné podmienky, ktoré pre daný útvar musia platiť. \\

Implementáciu algoritmického prehľadávania potenciálnych riešení pre vybrané otvorené problémy popíšeme v kapitole \ref{kap:implementation}. \\

V kapitole Záver zhrnieme dosiahnuté výsledky z oblasti magických útvarov a vyslovíme hypotézy, ktoré bude možné skúmať v budúcnosti.
