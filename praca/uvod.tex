\chapter*{Úvod} % chapter* je necislovana kapitola
\addcontentsline{toc}{chapter}{Úvod} % rucne pridanie do obsahu
\markboth{Úvod}{Úvod} % vyriesenie hlaviciek

Magické útvary zaujímali ľudí už odpradávna. Najznámejším z nich je magický štvorec. Ide o štvorcovú tabuľku veľkosti $3 \times 3$ vyplnenú číslami, pre ktorú platí, že súčty čísel v riadkoch, stĺpcoch a na oboch diagonálach sú rovnaké. Medzi ďalšie známe magické útvary patria obdĺžniky, grafy, kruhy alebo hviezdy \cite{antimagic}. \\

Existujú aj iné magické vlastnosti, ktoré môžeme na útvaroch skúmať. V štvorci nás môže zaujímať súčin prvkov a nie ich súčet (vtedy ide o multiplikatívny štvorec). Alebo budeme brať do úvahy súčet prvkov aj súčet ich druhých mocnín (vtedy hovoríme~o bimagickom štvorci). \\

V kapitole \ref{kap:definitions} uvedieme základné pojmy pri práci s útvarmi. Sformulujeme teóriu, ktorá bude definovať všetky magické útvary a ich vlastnosti. Spomenieme aj súčasný stav problematiky. \\

O niektorých útvaroch s konkrétnymi magickými vlastnosťami stále nevieme povedať, či existujú. V kapitole \ref{kap:openproblems} podrobnejšie preskúmame niektoré známe otvorené problémy z oblasti magických útvarov. Dokážeme niekoľko viet, na základe ktorých budeme vedieť implementovať algoritmy v kapitole \ref{kap:implementation}. \\

Magické vlastnosti niektorých útvarov ešte stále neboli preskúmané. Na nové definované problémy z oblasti magických útvarov sa pozrieme v kapitole \ref{kap:newproblems}. Zadefinujeme tieto nové typy útvarov: vrcholovo a hranovo bimagické grafy, vrcholovo a hranovo multiplikatívne magické grafy, bimagické a multiplikatívne magické obdĺžniky. Ku každému z nich uvedieme zistenia a nutné podmienky, ktoré pre daný útvar musia platiť. \\

Implementáciu algoritmického prehľadávania potenciálnych riešení pre vybrané otvorené problémy popíšeme v kapitole \ref{kap:implementation}. Program s jednotlivými algoritmami budeme písať v jazyku Python. \\

V kapitole Záver zhrnieme naše dosiahnuté výsledky z oblasti magických útvarov a vyslovíme hypotézy, ktoré bude možné skúmať v budúcnosti.
