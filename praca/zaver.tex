\chapter*{Záver}  % chapter* je necislovana kapitola
\addcontentsline{toc}{chapter}{Záver} % rucne pridanie do obsahu
\markboth{Záver}{Záver} % vyriesenie hlaviciek

Okrem toho sme objavili sme dva nové parametrické vzorce pre magický štvorec veľkosti $3 \times 3$, ktorého aspoň $6$ prvkov sú druhými mocninami kladných celých čísel. \\

Definovali sme normálne formy bimagických útvarov, ktoré vznikli z ich uzáverových vlastností na konštantný posun a kladný celočíselný násobok. Na základe toho sme popísali implementáciu algoritmického prehľadávania bimagických štvorcov veľkosti $5 \times 5$. \\

Impelmetovali sme aproximačný algoritmus na hľadanie multiplikatívnych magických štvorcov veľkosti $6 \times 6$, ktorý fungoval na princípe náhodného vzorkovania a jeho následnom optimalizovaní. Tým sa nám podarilo zostrojiť multiplikatívny štvorec s nízkym magickým variačným rozpätím $26$. \\

Pre vrcholovo bimagické grafy sme dokázali podmienky pre stupne vrcholov $1, 2$ a $3$. Objavili sme spôsob, ako takýto graf s mostom rozdeliť na dva podgrafy, ktoré majú tiež vrcholovo bimagické ohodnotenie. Spísali sme tri nutné podmienky, ktoré musí vrcholovo bimagický graf spĺňať. Zistili sme, že jediným kubickým grafom tohto typu je $K_{3,3}$. Podarilo sa nám implementovať algoritmus, ktorý skonštruuje kompletné bipartitné vrcholovo bimagické alebo supermagické grafy s daným počtom vrcholov. \\

Dokázali sme, že minimálny stupeň vrchola v hranovo bimagickom grafe je $3$, ale kubické grafy tohto typu neexistujú. Uviedli sme dolný odhad počtu hrán vzhľadom na počet vrcholov ???. \\

Zistili sme, že mnohé vlastnosti vrcholovo bimagických grafov majú aj vrcholovo multiplikatívne magické grafy. Objavili sme jediný strom s touto vlastnosťou. Ukázali sme neexistenciu superbimagických grafov tohto typu. \\

Na základe toho sme vyslovili hypotézu, že bimagické ani multiplikatívne magické obdĺžniky veľkosti $m \times n$ pre $m \neq n$ neexistujú. \\
