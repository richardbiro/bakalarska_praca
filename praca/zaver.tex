\chapter*{Záver}  % chapter* je necislovana kapitola
\addcontentsline{toc}{chapter}{Záver} % rucne pridanie do obsahu
\markboth{Záver}{Záver} % vyriesenie hlaviciek

V tejto práci sme skúmali magické útvary. Zaoberali sme sa známymi aj novými otvorenými problémami. Výsledkom sú zistenia, ktoré prinášajú pokrok v tejto oblasti. \\

Vymysleli sme parametrický vzorec generujúci štyri magické štvorce, ktorých aspoň $5$ prvkov sú druhými mocninami kladných celých čísel. Okrem toho sme objavili dva nové špecifické parametrické vzorce, pri ktorých bolo aspoň $6$ prvkov druhými mocninami kladných celých čísel. Algoritmické prehľadávanie nenašlo žiadne nové riešenie so $7$ prvkami. \\

Odvodili sme normálne formy bimagických štvorcov, ktoré vznikli z ich uzáverových vlastností na konštantný posun a kladný celočíselný násobok. Na základe toho sme popísali implementáciu algoritmického prehľadávania bimagických štvorcov veľkosti $5 \times 5$. Výsledkom tohto prehľadávania boli štyri magické štvorce, ktoré mali iba tri zlé bimagické súčty. \\

Impelmentovali sme aproximačný algoritmus na hľadanie multiplikatívnych magických štvorcov veľkosti $6 \times 6$, ktorý fungoval na princípe náhodného vzorkovania a jeho následnom optimalizovaní. Podarilo sa nám zostrojiť multiplikatívny štvorec s nízkym magickým variačným rozpätím $26$. \\

Pre vrcholovo bimagické grafy sme dokázali podmienky pre stupne vrcholov $1, 2, 3$. Objavili sme spôsob, ako takýto graf s mostom rozdeliť na dva podgrafy, ktoré majú tiež vrcholové bimagické ohodnotenie. Určili sme tri nutné podmienky, ktoré musí vrcholovo bimagický graf spĺňať pre ľubovoľnú dvojicu vrcholov. Zistili sme, že jediný kubický graf tohto typu je $K_{3,3}$. Podarilo sa nám implementovať algoritmus, ktorý zostrojí kompletné bipartitné vrcholovo bimagické alebo superbimagické grafy s daným počtom vrcholov. \\

Dokázali sme, že pre ľubovoľné dva susedné vrcholy v hranovo bimagickom grafe platí, že oba sú stupňa aspoň $3$ a zároveň jeden z nich je stupňa aspoň $4$ (okrem cesty na dvoch vrcholoch). Odvodili sme, že hranovo bimagický graf $G$ s číslom nezávislosti $\alpha (G)$ musí mať aspoň $2 V(G) - \frac{\alpha (G)}{2}$ hrán. Ukázali sme, že $K_{8,8}$ je hranovo superbimagický na základe už existujúceho superbimagického štvorca veľkosti $8 \times 8$. Uviedli sme hypotézu, ktorá hovorí, že všetky hranovo bimagické grafy sú nutne kompletné bipartitné alebo kompletné bipartitné bez jednej hrany. \\

Mnohé vlastnosti vrcholovo bimagických grafov zdedili aj vrcholovo multiplikatívne magické grafy. Objavili sme jediný vrcholovo multiplikatívny magický strom. Ukázali sme konštrukciu kompletných bipartitných a neexistenciu supermagických grafov tohto typu. Vyslovili sme hypotézu, že všetky vrcholovo bimagické aj multiplikatívne magické grafy sú nutne kompletné bipartitné. \\

Pre hranovo multiplikatívne magické grafy sme ukázali, že ich minimálny stupeň vrchola je $3$. Tieto typy grafov zostávajú predmetom ďalšieho možného skúmania. \\

Definovali sme pojem bimagického a multiplikatívneho magického obdĺžnika. Bimagické obdĺžniky sme previedli do normálnej formy, v ktorej bol ich najmenší prvok $1$. Pri multiplikatívnych magických sme obmedzili počet možných prvkov vzhľadom na veľkosť a najväčší prvok útvaru. Prehľadávaním sme nenašli žiadne riešenia pre veľkosti $3 \times n$ ani $4 \times n$. V oboch prípadoch sa nám podarilo nájsť iba čiastočné obdĺžniky, v ktorých nemali niektoré riadky hľadanú vlastnosť. Na základe toho sme vyslovili hypotézu, že bimagické ani multiplikatívne magické obdĺžniky veľkosti $m \times n$ pre $m \neq n$ neexistujú. V budúcnosti je možné preskúmať väčšie magické obdĺžniky (predovšetkým veľkosti $5 \times 6$) a pokúsiť sa nájsť vyhovujúci útvar s bimagickou alebo multiplikatívnou vlastnosťou. \\

Spravili sme prehľad v oblasti klasických magických útvarov. Sformulovali sme analogické problémy pre magické grafy. Vybrali sme si niekoľko známych aj nových otvorených problémov a implementovali sme algoritmické prehľadávanie priestoru ich potenciálnych riešení spojené s teoretickou analýzou. Vyslovili sme niekoľko hypotéz, ktoré bude možné skúmať v budúcnosti. Všetky ciele práce boli splnené.