\chapter{Výsledky}

\label{kap:results} % id kapitoly pre prikaz ref

V tejto kapitole analyzujeme výsledky algoritmického prehľadávania potenciálnych riešení pre vybrané otvorené problémy. \\

\section{Magické štvorce}

\subsection{Magické štvorce druhého stupňa}

\begin{result} Pre $u_1, v_1, u_2, v_2 < 1000$ dokážu parametrické vzorce vygenerovať iba jeden magický štvorec veľkosti $3 \times 3$, ktorého aspoň $7$ prvkov sú druhé mocniny kladných celých čísel (ten, ktorý poznáme). Dosiahneme ho napr. pre $u_1 = 3, v_1 = 4, u_2 = 2, v_2 = 9$ a vydelením prvkov ich spoločným deliteľom.
\end{result}

\begin{result} Pre $x = 1$ dostaneme štvorec, ktorého prvky nie sú navzájom rôzne. Pre $1 < x < 10^8$ nedokážu parametrické vzorce vygenerovať magický štvorec veľkosti $3 \times 3$, ktorého aspoň $7$ prvkov sú druhé mocniny prirodzených čísel.
\end{result}

\subsection{Bimagické štvorce}

\begin{result} Pre $h < 12500$ neexistuje bimagický štvorec veľkosti $5 \times 5$. Podarilo sa nájsť štyri magické štvorce veľkosti $5 \times 5$ so zápornými prvkami, ktoré majú iba $3$ zlé bimagické súčty:
\end{result}

\begin{center}
$\begin{array}{ |c|c|c|c|c| }
\hline
58 & 30 & -10 & -232 & -76 \\ 
\hline
-234 & -80 & 44 & 26 & 14  \\ 
\hline
160 & -18 & -230 & -74 & -68 \\ 
\hline
-198 & 66 & 48 & -12 & -134 \\ 
\hline
-16 & -228 & -82 & 62 & 34 \\ 
\hline
\end{array}$

$\begin{array}{ |c|c|c|c|c| } 
\hline
58 & 30 & -10 & -232 & -76 \\ 
\hline
-234 & -80 & 44 & 26 & 14  \\ 
\hline
96 & -18 & -230 & -74 & -4 \\ 
\hline
-134 & 66 & 48 & -12 & -198 \\ 
\hline
-16 & -228 & -82 & 62 & 34 \\ 
\hline
\end{array}$

$\begin{array}{ |c|c|c|c|c| }
\hline
58 & 30 & -10 & -232 & -76 \\ 
\hline
14 & -80 & 44 & 26 & -234 \\ 
\hline
-88 & -18 & -230 & -74 & 180 \\ 
\hline
-198 & 66 & 48 & -12 & -134 \\ 
\hline
-16 & -228 & -82 & 62 & 34 \\ 
\hline
\end{array}$

$\begin{array}{ |c|c|c|c|c| } 
\hline
58 & 30 & -10 & -232 & -76 \\ 
\hline
14 & -80 & 44 & 26 & -234 \\ 
\hline
-152 & -18 & -230 & -74 & 244 \\ 
\hline
-134 & 66 & 48 & -12 & -198 \\ 
\hline
-16 & -228 & -82 & 62 & 3 \\ 
\hline
\end{array}$
\end{center}



\subsection{Multiplikatívne magické štvorce}

\begin{result} Aproximačná metóda vzorkovaním nenašla žiaden multiplikatívny magický štvorec veľkosti $6 \times 6$ pre nízku prvočíselnú hranicu (v našom prípade sme si zvolili $h = 17$). Nasledovný multiplikatívny štvorec mal najmenšie rozpätie súčtov $26$:
\end{result}

\begin{center}
$\begin{array}{ |c|c|c|c|c|c| } 
\hline
150 & 384 & 297 & 78 & 308 & 340 \\ 
\hline
352 & 102 & 120 & 220 & 351 & 420 \\ 
\hline
330 & 252 & 286 & 450 & 136 & 96 \\ 
\hline
459 & 300 & 192 & 336 & 110 & 143 \\ 
\hline
156 & 121 & 140 & 306 & 480 & 360 \\ 
\hline
112 & 390 & 510 & 176 & 180 & 198 \\ 
\hline
\end{array}$
\end{center}


\section{Magické grafy}

\subsection{Vrcholovo bimagické grafy}
 
\begin{result} jediné súvislé grafy s menej ako $10$ vrcholmi, ktoré spĺňajú všetky podmienky (a teda môžu byť vrcholovo bimagickými), sú $K_{2,3}, K_{2,4}, K_{2,5}, K_{2,6}, K_{2,7}, K_{3,3}, \\
K_{3,4}, K_{3,5}, K_{3,6}, K_{4,4}, K_{4,5}, K_{2,3,3}, K_{2,3,4}$ a $K_{3,3,3}$.
\end{result}

Vieme, že $K_{i,j}$ je vrcholovo bimagický pre $i,j \geq 2, (i,j) \neq (2,2)$. Môžeme sa ľahko presvedčiť, že aj zvyšné grafy majú vrcholové bimagické ohodnotenie:
\begin{gather*}
K_{2,3,3} \rightarrow 11, 13 ~|~ 1, 8, 15 ~|~ 3, 5, 16 \\
K_{2,3,4} \rightarrow 11, 19 ~|~ 1, 9, 20 ~|~ 1, 2, 6, 21 \\
K_{3,3,3} \rightarrow 1, 12, 14 ~|~ 2, 9, 16 ~|~ 4, 6, 17
\end{gather*}

\section{Magické obdĺžniky}

\subsection{Bimagické obdĺžniky}

\begin{result} Neexistuje bimagický obdĺžnik veľkosti $3 \times n$, ktorého prvky sú kladné celé čísla menšie ako $400$.
\end{result} 

\begin{result} Neexistuje bimagický obdĺžnik veľkosti $3 \times n$, ktorého súčet prvkov v riadku je menší ako $384$. Podarilo sa nájsť niekoľko magických obdĺžnikov veľkosti $3 \times 6$, $3 \times 8$ a $3 \times 10$ s bimagickými stĺpcami a jediným nebimagickým riadkom. Najmenší z nich má súčet v stĺpci rovný $144$:
\end{result}

\begin{center}
$\begin{array}{ |c|c|c|c|c|c| } 
\hline
1 & 3 & 88 & 8 & 93 & 95 \\ 
\hline
63 & 56 & 51 & 91 & 11 & 16 \\ 
\hline
80 & 85 & 5 & 45 & 40 & 33 \\ 
\hline
\end{array}$
\end{center}


\subsection{Multiplikatívne magické obdĺžniky}

\begin{result} Neexistuje multiplikatívny magický obdĺžnik veľkosti $3 \times n$, ktorého súčet prvkov v riadku je menší ako $4000$. Podarilo sa nájsť niekoľko multiplikatívnych obdĺžnikov veľkosti $3 \times 6$ a $3 \times 9$ s magickými stĺpcami. Najmenší z nich má súčet v stĺpci rovný $485$:
\end{result}

\begin{center}
$\begin{array}{ |c|c|c|c|c|c| } 
\hline
14 & 294 & 16 & 385 & 60 & 396 \\ 
\hline
231 & 15 & 154 & 72 & 392 & 40 \\ 
\hline
240 & 176 & 315 & 28 & 33 & 49 \\ 
\hline
\end{array}$
\end{center}



