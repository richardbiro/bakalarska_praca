\chapter{Implementácia algoritmov}

\label{kap:implementation} % id kapitoly pre prikaz ref

Program sme písali v jazyku Python. Zvolili sme si ho predovšetkým z dôvodu, že má neobmedzené číselné premenné (niektoré algoritmy budú pracovať s hodnotami, ktoré sa nezmestia do bežnej $32$-bitovej premennej). Zdrojový kód je uvedený ako príloha k tejto práci.

\section{Magické štvorce}

\subsection{Magické štvorce druhého stupňa}

\begin{alg} 
\label{algsquare3x3uvw}
Vstupom sú navzájom rôzne kladné celé čísla $u_1, v_1, u_2, v_2 \in \mathbb{N}$. Výstupom je magický štvorec veľkosti $3 \times 3$, ktorého aspoň $7$ prvkov sú druhé mocniny kladných celých čísel. Algoritmus využije parametrický vzorec z vety \ref{3x3square5squares}, ktorý generuje vyhovujúce magické štvorce.
\end{alg}

\begin{algorithmic}
\State $p \gets (u_1^2 + v_1^2)(u_2^2 + 2u_2 v_2 - v_2^2)$
\State $q \gets (u_1^2 + 2u_1 v_1 - v_1^2)(u_2^2 + v_2^2)$
\State $r \gets (- u_1^2 + 2u_1 v_1 + v_1^2)(u_2^2 + v_2^2)$
\State $s \gets (u_1^2 + v_1^2)(-u_2^2 + 2u_2 v_2 + v_2^2)$
\State $t \gets (u_1^2 + v_1^2)(u_2^2 + v_2^2)$
\IF {aspoň dva z $3t^2 - p^2 - q^2, 3t^2 - p^2 - r^2, 3t^2 - q^2 - s^2, 3t^2 - r^2 - s^2$ sú štvorce}
    \STATE \textbf{return} prvý štvorec
\ENDIF
\IF {aspoň dva z $2(r^2 + s^2), 2(q^2 + s^2), 2(p^2 + r^2), 2(p^2 + q^2)$ sú štvorce}
    \STATE \textbf{return} druhý štvorec
\ENDIF
\IF {aspoň dva z $3t^2 - p^2 - q^2, r^2 + s^2 - p^2, p^2 + q^2 - s^2, 3t^2 - r^2 - s^2$ sú štvorce}
    \STATE \textbf{return} tretí štvorec
\ENDIF
\IF {aspoň dva z $3t^2 - p^2 - r^2, q^2 + s^2 - p^2, p^2 + r^2 - s^2, 3t^2 - q^2 - s^2$ sú štvorce}
    \STATE \textbf{return} štvrtý štvorec
\ENDIF
\end{algorithmic}


\begin{alg}
\label{algsquare3x3x}
Na vstupe dostaneme kladné celé číslo $x \in \mathbb{N}$. Výstupom je magický štvorec veľkosti $3 \times 3$, ktorého aspoň $7$ prvkov sú druhé mocniny kladných celých čísel. Algoritmus využije dva parametrické vzorce z vety \ref{3x3square6squares}, ktoré generujú vyhovujúce magické štvorce.
\end{alg}

\begin{algorithmic}
\STATE $x_1 \gets 8x^8 - 49x^6 + 6x^4 - 16x^2 + 2$
\STATE $x_2 \gets 8x^8 - x^6 + 30x^4 - 40x^2 + 2$
\STATE $x_3 \gets 8x^8 - 25x^6 + 18x^4 - 28x^2 + 2$
\IF {$x_1(x^2 - 2)$ je štvorec}
    \STATE \textbf{return} prvý štvorec
\ENDIF
\IF {$x_2(x^2 - 2)$ je štvorec}
    \STATE \textbf{return} prvý štvorec
\ENDIF
\IF {$x_3(x^2 - 2)$ je štvorec}
    \STATE \textbf{return} prvý štvorec
\ENDIF
\IF {$\frac{4x^{10} - 31x^8 + 76x^6 + 76x^4 - 31x^2 + 4}{2}$ je štvorec}
    \STATE \textbf{return} druhý štvorec
\ENDIF
\IF {$\frac{4x^{10} + 17x^8 + 4x^6 + 4x^4 + 17x^2 + 4}{2}$ je štvorec}
    \STATE \textbf{return} druhý štvorec
\ENDIF
\IF {$\frac{4x^{10} + 65x^8 - 68x^6 - 68x^4 + 65x^2 + 4}{2}$ je štvorec}
    \STATE \textbf{return} druhý štvorec
\ENDIF
\end{algorithmic}

\subsection{Bimagické štvorce}

\begin{alg}
\label{algsquare5x5b}
Na vstupe dostaneme kladné celé číslo $h \in \mathbb{N}$. Výstupom je bimagický štvorec veľkosti $5 \times 5$ s potenciálne zápornými prvkami. Algoritmus predpokladá, že magický súčet je rovný prostrednému prvku $s$ (konštrukciou z vety \ref{5x5bimagic1}). Potom si vygeneruje trojice $(a,b,c)$, z ktorých podľa vety \ref{5x5bimagic2} vytvorí štvorice $(-a+b+c, a-b+c, a+b-c, -a-b-c)$. Pokračuje hľadaním všetkých vyhovujúcich $s$ podľa vety \ref{5x5bimagic3} pre každý riadok štvorca. Na záver sa pokúsi doplniť vyhovujúce čísla do stĺpcov, a tým vygenerovať bimagický štvorec veľkosti $5 \times 5$.
\end{alg}

\begin{algorithmic}
\STATE $D_1 \gets dict()$
\STATE $D_2 \gets dict()$
\FORALL {$a,b,c \in \mathbb{N}; a < b < c; a^2 + b^2 + c^2 < h$}
    \STATE pridaj $(a,b,c)$ do $D_1[a^2 + b^2 + c^2]$
\ENDFOR
\FORALL {$k \in D_1$}
    \FORALL {$(a,b,c), (d,e,f), (g,h,i) \in D_1[k]$}
	  \STATE $diagonala1 \gets \{a+b-c, a-b+c, -a+b+c, -a-b-c\}$
	  \STATE $stred \gets \{d+e-f, d-e+f, -d+e+f, -d-e-f\}$
	  \STATE $diagonala2 \gets \{g+h-i,g-h+i,-g+h+i,-g-h-i\}$
          \FORALL {$p \in diagonala1, q \in stred, r \in diagonala2$}
		\STATE faktorizáciou nájdi všetky $s,n \in \mathbb{Z}$, pre ktoré platí
				\begin{gather*}
				(s+n+p+q+r)(s-n+p+q+r) = 4(pq + pr + qr + p^2 + q^2 + r^2 - \frac{k}{2})
				\end{gather*}
		\STATE pre každé dopočítaj $x \gets \frac{s - (p+q+r) \pm n}{2}, y \gets s - x - (p+q+r)$
		\IF {$x \in \mathbb{Z} ~\textbf{and}~ diagonala1, stred, diagonala2, \{x,y,s\}$ sú disjunktné}
			\STATE pridaj $(diagonala1.index(p),stred.index(q),diagonala2.index(r), p, q, r, x, y)$ do $D_2[s]$
		\ENDIF
	  \ENDFOR
    \ENDFOR
\ENDFOR
\FORALL {$k \in D_2$}
    \FORALL {$(pi_n, qi_n, ri_n, p_n, q_n, r_n, x_n, y_n) \in D_2[k], n \in \{1,2,3,4\}$}
          \IF {$\{pi_1, pi_2, pi_3, pi_4\} = \{qi_1, qi_2, qi_3, qi_4\} = \{ri_1, ri_2, ri_3, ri_4\} = \{0, 1, 2, 3\}$}
		\STATE skonštruuj nasledovný štvorec $A$
		\begin{center}
		$\begin{array}{ |c|c|c|c|c| }
		\hline
		p_1 & x_1 & q_1 & y_1 & r_1 \\ 
		\hline
		x_2 & p_2 & q_2 & r_2 & y_2  \\ 
		\hline
		- & - & s & - & - \\ 
		\hline
		x_3 & r_3 & q_3 & p_3 & x_4 \\ 
		\hline
		r_4 & x_4 & q_4 & y_4 & p_4 \\
		\hline
		\end{array}$
		\end{center}

		\FORALL {$A^\prime; A^\prime = A$ ~\textbf{or}~ $A^\prime$ má vymenené niektoré $x_n, y_n$ oproti $A$}
			\STATE doplň čísla do prostredného riadku $A^\prime$ tak, aby vznikol magický štvorec
			\IF {$A^\prime$ je bimagický}
				\STATE \textbf{print}($A^\prime$)
			\ENDIF
		\ENDFOR
	  \ENDIF
    \ENDFOR
\ENDFOR
\end{algorithmic}

\subsection{Multiplikatívne magické štvorce} 

\begin{alg}
\label{algsquare6x6mm}
Na vstupe dostaneme kladné celé čísla $p, h$. Výstupom je multiplikatívny štvorec veľkosti $6 \times 6$, ktorý má čo najbližšie k magickej vlastnosti (odchýlky súčtov v riadkoch, stĺpcoch a diagonálach sú najmenšie možné), používa $p$ štvorcových vzoriek a žiadna z nich nemá na začiatku vyššiu hodnotu ako $h$. Tento aproximačný algoritmus využíva vetu \ref{addmultsquarepattern}. Začneme so štvorcom, ktorého všetky prvky sú $1$. Potom ho niekoľkokrát prenásobíme náhodnou štvorcovou vzorkou s náhodnou hodnotou. Nakoniec troma operáciami upravujeme hodnoty tak, aby sme sa čo najviac priblížili k multiplikatívnemu magickému štvorcu: vymeníme navzájom dve hodnoty, zmeníme jednu hodnotu alebo pripočítame k hodnotám jedno z čísel $-1, 0, 1$. Ako primárny indikátor sme si zvolili variačné rozpätie vzniknutého štvorca (rozdiel najväčšieho a najmenšieho magického súčtu) a ako sekundárny indikátor počet jeho rôznych magických súčtov. Čím majú oba indikátory menšiu hodnotu, tým je výsledok lepší.
\end{alg}

\begin{algorithmic}
\STATE $vzorky \gets$ [všetky vzorky v štvorci veľkosti $6 \times 6$ uložené ako šestice]
\STATE $stvorcove \gets $ []
\FORALL {$v_1,v_2,v_3,v_4,v_5,v_6 \in vzorky$}
    \IF {$v_1,v_2,v_3,v_4,v_5,v_6$ vypĺňajú celý štvorec}
	\STATE pridaj $(v_1,v_2,v_3,v_4,v_5,v_6)$ do $stvorcove$
    \ENDIF
\ENDFOR
\WHILE {true}
	\STATE $hodnoty \gets$ []
	\STATE $pouzite \gets$ []
	\FOR {$i \gets 0, i < p$}
		\STATE pridaj $6$ náhodných hodnôt medzi $1$ a $h$ do $hodnoty$
		\STATE pridaj náhodnú štvorcovú vzorku z $stvorcove$ do $pouzite$
	\ENDFOR
	\WHILE {true}
		\STATE $stvorec \gets$ štvorec veľkosti $6 \times 6$, ktorého prvky sú $1$
		\STATE $counter \gets 0$
		\FORALL {stvorcovaVzorka $\in pouzite$}
			\FORALL {$v \in $ stvorcovaVzorka}
				\STATE prenásob $stvorec$ vzorkou $v$ s hodnotou $hodnoty[counter]$
				\STATE counter++
			\ENDFOR
		\ENDFOR
		\IF {$stvorec$ obsahuje navzájom rôzne prvky}
			\STATE $stavZaciatok = stav$
			\STATE $stav \gets$ (variačné rozpätie $stvorec$, počet rôznych súčtov $stvorec$)
			\FORALL {$hodnotyNove$; $hodnotyNove$ dostanem z $hodnoty$ operáciou}
				\STATE $stvorecNovy \gets$ štvorec prenásobený vzorkami s novými hodnotami
				\STATE $stavNovy \gets$ (variačné rozpätie $stvorecNovy$, počet rôznych súčtov $stvorecNovy$)
				\IF {$stavNovy$ je lepší ako $stav$}
					\STATE $stav \gets stavNovy$
					\STATE $stvorec \gets stvorecNovy$
				\ENDIF
			\ENDFOR
			\IF {$stav$ nie je lepší ako $stavZaciatok$}
				\STATE \textbf{print}(stav,stvorec)
				\STATE \textbf{break}
			\ENDIF
		\ENDIF
	\ENDWHILE
\ENDWHILE
\end{algorithmic}

\section{Magické grafy}

\subsection{Vrcholovo bimagické grafy}

Algoritmy pracujú so súvislými grafmi s daným počtom vrcholov, pričom sú uložené v $graph6$ formáte. Na prácu s ním sme využili funkciu $read \_ graph6$ z knižnice $networkx$. Údaje sme získali z webstránky, ktorá obsahuje kolekciu grafov rôzneho druhu \cite{graphlist}. \\
 
\begin{alg}
\label{algvbgcondition}
Na vstupe dostaneme ľubovoľný súvislý graf. Výstupom je odpoveď, či daný graf môže byť vrcholovo bimagický. Pre každú dvojicu jeho vrcholov overíme, či spĺňa podmienky z vety \ref{vbgcondition}. Ak existuje dvojica vrcholov, pre ktorú graf nevyhovuje niektorej z troch podmienok, tak môžeme s istotou povedať, že nie je vrcholovo bimagický.
\end{alg}

\begin{algorithmic}
\FOR {$v_1, v_2 \in V(G)$}
    \STATE $x \gets |\{susedia[v1]\} - \{susedia[v2]\}|$
    \STATE $y \gets |\{susedia[v2]\} - \{susedia[v1]\}|$
    \IF {$xy = 0 ~\textbf{and}~ x+y > 0$}
	\STATE \textbf{return} $G$ nie je vrcholovo bimagický
    \ENDIF
    \IF {$x = 1 ~\textbf{or}~ y = 1$}
	\STATE \textbf{return} $G$ nie je vrcholovo bimagický
    \ENDIF
    \IF {$x = 2 ~\textbf{and}~ y = 2$}
	\STATE \textbf{return} $G$ nie je vrcholovo bimagický
    \ENDIF
\ENDFOR
\end{algorithmic}


\begin{alg}
\label{algvbgkij}
Na vstupe dostaneme čísla $i,j \in \mathbb{N}$. Výstupom má byť vrcholové bimagické ohodnotenie grafu $K_{i,j}$. Algoritmus bude replikovať indukčný dôkaz vety \ref{vbgkij}.
\end{alg}

\begin{algorithmic}
\IF {$i > j$}
	\STATE \textbf{return} ohodnot($j,i$)
\ENDIF
\IF {$i \leq 1 ~\textbf{or}~ (i = 2 ~\textbf{and}~ j = 2)$}
	\STATE \textbf{return} nedá sa ohodnotiť
\ENDIF
\IF {$i = 2$}
	\STATE \textbf{return} $(\frac{j(j-1)}{2} + 1, \frac{j(j-1)(3j^2-7j+14)}{24}), (1, \dots , j-1, \frac{j(j-1)(3j^2-7j+14)}{24} + 1)$
\ENDIF
\IF {$i = 3$}
	\STATE \textbf{return} $(1, \frac{j(j+1)}{2} - 1, \frac{j(j+1)(3j^2-j-14)}{24} + 1), (2, \dots , j, \frac{j(j+1)(3j^2-j-14)}{24} + 2)$
\ENDIF
\IF {$i = 4 ~\textbf{and}~ j = 4$}
	\STATE \textbf{return} $(1, 4, 6, 7), (2, 3, 5, 8)$
\ENDIF
\IF {$i = 4 ~\textbf{and}~ j = 5$}
	\STATE \textbf{return} $(2, 12, 13, 15), (1, 4, 8, 10, 19)$
\ENDIF
\STATE $H \gets$ ohodnot($i - 2, j - 3$)
\STATE $m \gets$ max($H$) + 1
\STATE na ľavú stranu $H$ pridaj $4m, 5m$
\STATE na pravú stranu $H$ pridaj $m, 2m, 6m$
\STATE \textbf{return} $H$
\end{algorithmic}

\begin{alg}
\label{algvsbgkij}
Na vstupe dostaneme číslo $n \in \mathbb{N}$. Výstupom algoritmu má byť vrcholové superbimagické ohodnotenie kompletného bipartitného grafu s $n$ vrcholmi. Algoritmus bude replikovať indukčný dôkaz vety \ref{vsbgkij}.
\end{alg}

\begin{algorithmic}
\IF {$n < 7$}
	\STATE \textbf{return} nedá sa ohodnotiť
\ENDIF
\IF {$n ~\textbf{mod}~ 4 = 1 ~\textbf{or}~ n ~\textbf{mod}~ 4 = 2$}
	\STATE \textbf{return} nedá sa ohodnotiť
\ENDIF
\IF {$n = 7$}
	\STATE \textbf{return} $(1, 2, 4, 7), (3, 5, 6)$
\ENDIF
\IF {$n = 8$}
	\STATE \textbf{return} $(1, 4, 6, 7), (2, 3, 5, 8)$
\ENDIF
\IF {$n = 11$}
	\STATE \textbf{return} $(1, 3, 4, 5, 9, 11), (2, 6, 7, 8, 10)$
\ENDIF
\IF {$n = 12$}
	\STATE \textbf{return} $(1, 3, 7, 8, 9, 11), (2, 4, 5, 6, 10, 12)$
\ENDIF
\STATE $H \gets$ ohodnot($n - 8$)
\FOR {$x \gets 1, 8$}
	\IF {$x \in \{1,4,6,7\}$}
		\STATE pridaj $(n-8)+x$ na ľavú stranu H
	\ELSE
		\STATE pridaj $(n-8)+x$ na pravú stranu H
	\ENDIF
\ENDFOR
\STATE \textbf{return} $H$
\end{algorithmic}

\subsection{Vrcholovo multiplikatívne magické grafy}

\begin{alg}
\label{algvmmgkij}
Na vstupe dostaneme kompletný bipartitný graf $K_{i,j}$. Výstupom má byť vrcholové multiplikatívne magické ohodnotenie tohto grafu. Algoritmus bude replikovať indukčný dôkaz vety \ref{vmmgkij}.
\end{alg}

\begin{algorithmic}
\IF {$i > j$}
	\STATE \textbf{return} ohodnot(j,i)
\ENDIF
\IF {$i \leq 1 ~\textbf{or}~ (i = 2 ~\textbf{and}~ j = 2)$}
	\STATE \textbf{return} nedá sa ohodnotiť
\ENDIF
\IF {$i = 2 ~\textbf{and}~ j = 3$}
	\STATE \textbf{return} $(5, 12), (1, 6, 10)$
\ENDIF
\IF {$i = 2 ~\textbf{and}~ j = 4$}
	\STATE \textbf{return} $(9, 16), (1, 2, 4, 18)$
\ENDIF
\IF {$i = 2$}
	\STATE \textbf{return} $((j-1)! + 1, (j-1)! [(j-1)! + 1 - \frac{j(j-1)}{2}] ), (1, \dots , j-1, [(j-1)! + 1] [(j-1)! + 1 - \frac{j(j-1)}{2}] )$
\ENDIF
\IF {$i = 3$}
	\STATE \textbf{return} $(1, j! + 1, j! (j! + 3 - \frac{j(j+1)}{2}) ), (2, \dots , j, (j! + 1) (j! + 3 - \frac{j(j+1)}{2}))$
\ENDIF
\IF {$i = 4 ~\textbf{and}~ j = 4$}
	\STATE \textbf{return} $(1, 5, 6, 12), (2, 3, 4, 15)$
\ENDIF
\IF {$i = 4 ~\textbf{and}~ j = 5$}
	\STATE \textbf{return} $(2, 10, 20, 27), (1, 3, 6, 24, 25)$
\ENDIF
\STATE $H \gets$ ohodnot($i - 2, j - 3$)
\STATE $x \gets$ max($H$) + 1
\STATE $y \gets$ max($H$) + 2
\STATE na ľavú stranu $H$ pridaj $2xy, 2xy - x - y$
\STATE na pravú stranu $H$ pridaj $2(2xy - x - y), x, y$
\STATE \textbf{return} $H$
\end{algorithmic}

\section{Magické obdĺžniky}

Všetky algoritmy pracujú efektívne s poradím stĺpcov podľa dôsledku \ref{rectangleorder}.

\subsection{Bimagické obdĺžniky}

\begin{alg}
\label{algbos}
Na vstupe dostaneme kladné celé čísla $n,s \in \mathbb{N}, n \geq 4$. Výstupom má byť bimagický obdĺžnik veľkosti $3 \times n$, ktorého prvky sú kladné celé čísla, pričom ich súčet v každom stĺpci je $s$. Náš algoritmus predpokladá, že najmenší prvok obdĺžnika je $1$ (s využitím vety \ref{rectangle1}). Pre každú trojicu rôznych celých čísel $a,b,c$ so súčtom $s$ si predpočíta ich bimagický súčet. Keďže platí podmienka z vety \ref{rectangle1cond}, vieme nájsť celé čísla $d,e$ tak, aby mohli byť trojice $(a,b,c)$ a $(1,d,e)$ použité ako stĺpce v tom istom bimagickom obdĺžniku. Pre každú takú trojicu $(a,b,c)$ si algoritmus uloží hodnoty $(1,d,e)$ ako kľúč do asociatívneho poľa. Potom toto pole prejde a v každom kľúči vyberie $n-1$ rôznych zapamätaných trojíc (ku ktorým pridá trojicu v kľúči).
\end{alg}

\begin{algorithmic}
\STATE $D \gets dict()$
\FOR {$a \gets 2, \ceil{\frac{s}{3}}$}
	\FOR {$b \gets a+1, \ceil{\frac{s-a}{2}}$}
		\STATE $c \gets s-a-b$
		\STATE $t \gets a^2+b^2+c^2$
		\IF {$2t - (s-1)^2 - 2$ je štvorec}
			\STATE pridaj $(a,b,c)$ do $D[(1, \frac{s-1 + \sqrt{2t - (s-1)^2 - 2}}{2}, \frac{s-1 - \sqrt{2t - (s-1)^2 - 2}}{2})]$
		\ENDIF
	\ENDFOR
\ENDFOR
\FORALL {$k \in D$}
	\FORALL {$(a_1, b_1, c_1), \dots , (a_{n-1}, b_{n-1}, c_{n-1}) \in D[k]$}
		\IF  {$1, k[1], k[2], a_1, b_1, c_1, \dots , a_{n-1}, b_{n-1}, c_{n-1}$ sú navzájom rôzne}
			\FORALL {permutácie $(a_i, b_i, c_i), i \in \{1, \dots , n-1\}$}
				\STATE vytvor obdĺžnik s prvým stĺpcom $1, k[1], k[2]$ a $j$-tym stĺpcom $a_{j-1}, b_{j-1}, c_{j-1}$ pre $j \in \{2, \dots , n\}$
				\IF {obdĺžnik má bimagické riadky}
					\STATE \textbf{print}(obdlznik)
				\ENDIF
			\ENDFOR
		\ENDIF
	\ENDFOR
\ENDFOR
\end{algorithmic}

\begin{alg}
\label{algboh}
Na vstupe dostaneme kladné celé čísla $n,h \in \mathbb{N}, n \geq 4$. Výstupom má byť bimagický obdĺžnik veľkosti $3 \times n$, ktorého prvky sú kladné celé čísla neprevyšujúce $h$. Od predchádzajúceho algoritmu sa líši tým, že si do asociatívneho poľa ukladá trojice čísel od $2$ do $h$ a nie trojice s daným súčtom.
\end{alg}

\begin{algorithmic}
\STATE $D \gets dict()$
\FOR {$a \gets 2, h$}
	\FOR {$b \gets a+1, h$}
		\FOR {$c \gets b+1, h$} 
			\STATE $s \gets a+b+c$
			\STATE $t \gets a^2+b^2+c^2$
			\IF {$2t - (s-1)^2 - 2$ je štvorec}
				\STATE pridaj $(a,b,c)$ do $D[(1, \frac{s-1 + \sqrt{2t - (s-1)^2 - 2}}{2}, \frac{s-1 - \sqrt{2t - (s-1)^2 - 2}}{2})]$
			\ENDIF
		\ENDFOR
	\ENDFOR
\ENDFOR
\STATE pokračuj rovnako ako predchádzajúci algoritmus
\end{algorithmic}

Tento algoritmus je efektívnejší ako algoritmus \ref{algbos}, lebo spracováva všetky trojice a nie iba tie s konkrétnym súčtom. Jeho nevýhodou je obmedzená funkcionalita (z pamäťových dôvodov je nepoužiteľný pre $h > 500$) a zlá iterácia (ak poznáme riešenia pre $h = 400$, tak na nájdenie riešení pre $h = 450$ treba spustiť algoritmus úplne od začiatku).

%\begin{alg}
%\label{algrectangle0}
%Na vstupe dostaneme kladné celé čísla $n,h \in \mathbb{N}$. Výstupom má byť bimagický obdĺžnik veľkosti $3 \times n$, ktorého prvky sú celé (potenciálne záporné) čísla v absolútnej hodnote neprevyšujúce $h$. Náš algoritmus predpokladá, že bimagický obdĺžnik má v každom riadku aj stĺpci nulový súčet (s využitím vety \ref{rectangle0}). Trojica prvkov v každom stĺpci je preto v tvare $(a, b, -a-b)$. Pre každú dvojicu celých čísel $a,b$ (pričom aspoň jedno z nich je nepárne, čo zaručuje veta \ref{rectangleodd}) si algoritmus uloží hodnotu výrazu $a^2 + b^2 + (-a-b)^2$ ako kľúč do asociatívneho poľa. Potom toto pole prejde a v každom kľúči vyberie $n$ rôznych zapamätaných dvojíc $(a,b)$, z ktorých si spätne zrekonštruuje trojice $(a,b,-a-b)$.
%\end{alg}

%\begin{algorithmic}
%\STATE $D \gets dict()$
%\FOR {$a \gets 0, h$}
%	\FORALL {$b \in \{-a+1, -a, \dots , a-1\}; ab ~\textbf{mod}~ 2 = 0$}
%		\STATE $t \gets a^2 + b^2 + (-a-b)^2$
%		\STATE pridaj $(a,b)$ do $D[t]$
%	\ENDFOR
%\ENDFOR
%\FORALL {$k \in D$}
%	\FORALL {$(a_1, b_1), \dots , (a_n, b_n) \in D[k]$}
%		\IF  {$a_1, b_1, - a_1 - b_1, \dots , a_n, b_n, - a_n - b_n$ sú navzájom rôzne}
%			\FORALL {permutácie $(a_i, b_i, - a_i - b_i), i \in \{2, \dots , n\}$}
%				\STATE vytvor obdĺžnik s $j$-tym stĺpcom $a_j, b_j, - a_j - b_j$ pre $j \in \{1, \dots , n\}$
%				\IF {obdĺžnik má bimagické riadky}
%					\STATE \textbf{print}(obdlznik)
%				\ENDIF
%			\ENDFOR
%		\ENDIF
%	\ENDFOR
%\ENDFOR
%\end{algorithmic}


\subsection{Multiplikatívne magické obdĺžniky}

\begin{alg}
\label{algmmos}
Na vstupe dostaneme čísla $n,s \in \mathbb{N}, n \geq 4$. Výstupom má byť multiplikatívny magický obdĺžnik veľkosti $3 \times n$, ktorého prvky sú kladné celé čísla, pričom ich súčet v každom stĺpci je $s$. Vieme, že obdĺžnik nemôže obsahovať číslo $x$, pre ktoré neplatí veta \ref{rectanglemax}. Náš algoritmus si pre každú trojicu vyhovujúcich rôznych kladných čísel uloží ich súčin ako kľúč do asociatívneho poľa. Potom toto pole prejde a v každom kľúči vyberie $n$ rôznych zapamätaných trojíc.
\end{alg}

\begin{algorithmic}
\STATE $D \gets dict()$
\STATE $vyhovuju \gets \{x ~|~ x \in \{1, \dots , s\}, x$ nie je prvočíslo alebo $xn \leq s\}$
\FORALL {$a,b \in vyhovuju; a < b; a + 2b < s$}
	\STATE $c \gets s-a-b$
	\IF {$c \in vyhovuju$}
		\STATE $p \gets abc$
		\STATE pridaj $(a,b,c)$ do $D[p]$
	\ENDIF
\ENDFOR
\FORALL {$k \in D$}
	\FORALL {$(a_1, b_1, c_1), \dots , (a_n, b_n, c_n) \in D[k]$}
		\IF  {$a_1, b_1, c_1, \dots , a_n, b_n, c_n$ sú navzájom rôzne}
			\FORALL {permutácie $(a_i, b_i, c_i), i \in \{2, \dots , n\}$}
				\STATE vytvor obdĺžnik s $j$-tym stĺpcom $a_{j}, b_{j}, c_{j}$ pre $j \in \{1, \dots , n\}$
				\IF {obdĺžnik má multiplikatívne magické riadky}
					\STATE \textbf{print}(obdlznik)
				\ENDIF
			\ENDFOR
		\ENDIF
	\ENDFOR
\ENDFOR
\end{algorithmic}

%\begin{alg}
%\label{algmmoh}
%Na vstupe dostaneme čísla $n,h \in \mathbb{N}, n \geq 4$. Výstupom má byť multiplikatívny magický obdĺžnik veľkosti $3 \times n$, ktorého prvky sú kladné celé čísla neprevyšujúce $h$. Od predchádzajúceho algoritmu sa líši tým, že si do asociatívneho poľa ukladá trojice čísel od $1$ do $h$ a nie trojice s daným súčtom.
%\end{alg}

%\begin{algorithmic}
%\STATE $D \gets dict()$
%\STATE $vyhovuju \gets \{x ~|~ x \in \{1, \dots , h\}, x$ nie je prvočíslo alebo $xn \leq h\}$
%\FORALL {$a,b,c \in vyhovuju; a < b < c < h$}
%	\STATE $s \gets a+b+c$
%	\STATE $p \gets abc$
%	\STATE pridaj $(a,b,c)$ do $D[(s,p)]$
%\ENDFOR
%\STATE pokračuj rovnako ako predchádzajúci algoritmus
%\end{algorithmic}

