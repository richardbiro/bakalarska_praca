\chapter{Základné pojmy a definície}

\label{kap:definitions} % id kapitoly pre prikaz ref

V tejto kapitole uvedieme základné pojmy a definície pri práci s útvarmi, ako aj súčasný stav danej problematiky. \\

\textit{Priamka} je objekt v konečnom geometrickom systéme, ktorý prechádza aspoň jedným bodom. \textit{Útvar} je množina bodov, ktoré sú spojené priamkami. \textit{Prvok} je bod útvaru, ktorý má priradenú hodnotu $x$, kde $x$ je kladné celé číslo. Prvky útvaru majú priradené navzájom rôzne hodnoty. Útvar má \textit{magickú vlastnosť} ak všetky jeho priamky majú magickú vlastnosť. \\

Ak má útvar pravidelné usporiadanie, môže byť reprezentovaný maticou (priamkami budú riadky, stĺpce a prípadne diagonály danej matice) alebo neorientovaným ohodnoteným grafom.

\section{Magické útvary}
\begin{definition} Útvar je magický ak súčet prvkov na každej jeho priamke je konštantný.
\end{definition}

\subsection{Magické štvorce}
\begin{definition} Magický štvorec je matica prvkov veľkosti $n \times n$, pre ktorú platí, že súčet prvkov v každom riadku, stĺpci a na oboch diagonálach je konštantný.
\end{definition}

\begin{center}
$\begin{array}{ |c|c|c| } 
\hline
4 & 3 & 8 \\ 
\hline
9 & 5 & 1 \\ 
\hline
2 & 7 & 6 \\
\hline
\end{array}$
\end{center}

\begin{note} Ak je súčet prvkov v každom riadku a stĺpci konštantný, daný štvorec nazývame \textbf{semimagickým}. Ak je súčet na oboch diagonálach rovnaký, ale iný ako súčet v riadkoch a stĺpcoch, daný štvorec nazývame \textbf{panmagickým}.
\end{note}

Špeciálnu triedu tvoria magické štvorce, ktorých prvky sú $k$-tymi mocninami kladných celých čísel. Príklad štvorca pre $n = 4, k = 2$:

\begin{center}
$\begin{array}{ |c|c|c|c| } 
\hline
48^2 & 23^2 & 6^2 & 19^2 \\ 
\hline
21^2 & 26^2 & 33^2 & 32^2 \\ 
\hline
1^2 & 36^2 & 13^2 & 42^2 \\
\hline
22^2 & 27^2 & 44^2 & 9^2 \\
\hline
\end{array}$
\end{center}

Existencia štvorca pre $n = 3, k = 2$ je otvoreným problémom. Je dokázané, že ak by taký štvorec existoval, jeho prvky by museli byť väčšie ako $10^{16}$. Nikomu sa nepodarilo nájsť ani magický štvorec, ktorého $8$ prvkov sú druhé mocniny prirodzených čísel. A je známe iba jedno základné riešenie so $7$ prvkami, ktoré objavil v roku 1999 Andrew Bremner \cite{multimagie}.

\begin{center}
$\begin{array}{ |c|c|c| } 
\hline
373^2 & 289^2 & 565^2 \\ 
\hline
360721 & 425^2 & 23^2 \\ 
\hline
205^2 & 527^2 & 222121 \\
\hline
\end{array}$
\end{center}

\begin{note} Existujú vzorce, ktoré dokážu vygenerovať magický štvorec so $6$ prvkami, ktoré sú druhými mocninami prirodzených čísel.
\end{note}

Pre $n = k = 3$ je dokázané, že taký magický štvorec neexistuje. Existencia štvorcov pre $4 \leq n \leq 6, k = 3$ je otvoreným problémom. Pre $4 \leq n \leq 10, k \geq 4$ sú známe iba semimagické štvorce \cite{multimagie}. \\

\subsection{Magické obdĺžniky}
\begin{definition} Magický obdĺžnik je matica prvkov veľkosti $m \times n$, pre ktorú platí, že súčet prvkov v každom riadku je konštantný a zároveň súčet prvkov v každom stĺpci je konštantný.
\end{definition}

\begin{center}
$\begin{array}{ |c|c|c|c| } 
\hline
1 & 7 & 6 & 4 \\ 
\hline
8 & 2 & 3 & 5 \\
\hline
\end{array}$
\end{center}

Nevyžadujeme, aby boli súčty v riadkoch a stĺpcoch rovnaké, pretože pre $m \neq n$ vieme ľahko odvodiť, že by museli byť rovné $0$ (čo je spor s tým, že prvky sú navzájom rôzne kladné celé čísla). \\

Slovenský matematik Marián Trenkler skúmal obdĺžniky veľkosti $m \times n$, ktoré sú supermagické (ich prvkami sú čísla od $1$ po $mn$) \cite{rectangles}.

\begin{theorem} (Trenkler, 1999) Pre všetky prirodzené $n > 2$ vieme zostrojiť supermagický obdĺžnik veľkosti $2 \times (2n - 2)$ aj $n \times n^2$.
\end{theorem}

Keďže obdĺžniková matica nemá diagonály, pri definícii ich neuvažujeme. Z toho vyplýva, že v ľubovoľnom magickom obdĺžniku vieme vymeniť dva riadky alebo stĺpce a magická vlastnosť ostane zachovaná. \\

Semimagické štvorce sú špeciálnym prípadom magických obdĺžnikov pre $m = n$. \\

\subsection{Magické grafy}
\begin{definition} Magický graf je neorientovaný graf s ohodnotenými hranami, v ktorom pre každý vrchol platí, že súčet hrán incidentných s ním je konštantný. Vrcholy sú považované za prvky útvaru.
\end{definition}

Príklad magického grafu je na obrázku \ref{obr:fig_magic_graph}.

\begin{figure}[H]
\centerline{\includegraphics[width=0.4\textwidth]{images/magic_graph}}
\caption[Magický graf]{Magický graf s magickým súčtom 60 \cite{regular}}
\label{obr:fig_magic_graph}
\end{figure}

Slovenskí matematici Samuel Jezný a Marián Trenkler dokázali vetu, ktorá hovorí o tom, kedy je graf magický \cite{graphs}.

\begin{theorem} (Jezný, Trenkler, 1983) Graf je magický práve vtedy, keď každá hrana $G$ patrí do nejakého $(1-2)$-faktora a zároveň každá dvojica hrán $e_1, e_2$ je separovateľná $(1-2)$-faktorom grafu $G$.
\end{theorem}

\begin{note} $(1-2)$-faktor grafu je jeho rozklad na izolované hrany a kružnice.
\end{note} 

Magická vlastnosť grafu sa dá skúmať viacerými spôsobmi. Môžeme ohodnotiť vrcholy a pre každú hranu zrátať súčet hodnôt jej koncových vrcholov. Alebo pre každý vrchol zrátať súčet hodnôt jeho susedov. Ešte nikto neskúmal na grafoch bimagické a multiplikatívne magické vlastnosti (definované nižšie). \\



\section{Multiplikatívne útvary}
\begin{definition} Útvar je multiplikatívny ak súčin prvkov na každej jeho priamke je konštantný.
\end{definition}

\begin{center}
$\begin{array}{ |c|c|c| } 
\hline
8 & 256 & 2 \\ 
\hline
4 & 16 & 64 \\ 
\hline
128 & 1 & 32 \\
\hline
\end{array}$
\end{center}

\begin{note} Semimultiplikatívne a panmultiplikatívne štvorce sú definované analogicky.
\end{note}

K ľubovoľnému magickému štvorcu vieme zostrojiť multiplikatívny štvorec napríklad tak, že všetky jeho prvky $x$ nahradíme $2^x$. \\

Tieto typy štvorcov sa dajú hľadať vzorkovou metódou. Vzorku získame tak, že zvolíme niekoľko prvkov štvorca, pričom:
\begin{itemize}
\item v každom riadku je zvolený práve jeden prvok
\item v každom stĺpci je zvolený práve jeden prvok
\item na každej diagonále je zvolený práve jeden prvok
\end{itemize}

Princíp prehľadávania je potom jednoduchý. Najprv začneme so štvorcom, ktorého všetky prvky majú hodnotu $1$. Potom si opakovane vyberieme ľubovoľnú vzorku a všetky jej zvolené prvky vynásobíme nejakou konštantou. Tým generujeme štvorec, ktorý je multiplikatívny (za predpokladu, že výsledné prvky sú navzájom rôzne). \\

\section{Bimagické útvary}
\begin{definition} Útvar je bimagický ak je magický a umocnením každého jeho prvku na druhú dostaneme opäť magický útvar.
\end{definition}

Je zrejmé, že bimagický štvorec veľkosti $2 \times 2$ neexistuje. Edouard Lucas, Luke Pebody a Jean-Claude Rosa dokázali silnejšie tvrdenia \cite{multimagie}.

\begin{theorem} (Lucas, 1891) Neexistuje bimagický štvorec veľkosti $3 \times 3$.
\end{theorem}

\begin{theorem} (Pebody, Rosa, 2004) Neexistuje bimagický štvorec veľkosti $4 \times 4$.
\end{theorem}

Na to, aby bol štvorec veľkosti $5 \times 5$ bimagickým, muselo by byť jeho 12 magických a 12 bimagických súčtov rovnakých. V júni 2010 našiel Michael Quist čiastočné riešenie, ktoré obsahovalo 23 správnych súčtov \cite{multimagie}:

\begin{center}
$\begin{array}{ |c|c|c|c|c| }
\hline
25 & 129 & 200 & 295 & 195 \\ 
\hline
257 & 165 & 1 & 225 & 196  \\ 
\hline
127 & 340 & 171 & 111 & 95 \\ 
\hline
267 & 85 & 265 & 176 & 51 \\ 
\hline
168 & 125 & 207 & 37 & 307 \\
\hline
\end{array}$
\end{center}

Existencia riešenia pre $5 \times 5$ (ktoré by malo 24 správnych súčtov) je však dodnes otvoreným problémom. \\

V roku 2006 našiel Jaroslaw Wroblewski riešenie pre $6 \times 6$ \cite{multimagie}:

\begin{center}
$\begin{array}{ |c|c|c|c|c|c| } 
\hline
17 & 36 & 55 & 124 & 62 & 114 \\ 
\hline
58 & 40 & 129 & 50 & 111 & 20 \\ 
\hline
108 & 135 & 34 & 44 & 38 & 49 \\
\hline
87 & 98 & 92 & 102 & 1 & 28 \\
\hline
116 & 25 & 86 & 7 & 96 & 78 \\
\hline
22 & 74 & 12 & 81 & 100 & 119 \\
\hline
\end{array}$
\end{center}

Na tomto štvorci je zaujímavé to, že má asociatívnu vlastnosť - súčet protiľahlých prvkov je konštantný. \\

Georges Pfeffermann našiel v roku 1890 superbimagický štvorec veľkosti $8 \times 8$. Použil v ňom všetky čísla z množiny $\{1, 2, \dots , 64\}$ \cite{multimagie}: \\

\begin{center}
$\begin{array}{ |c|c|c|c|c|c|c|c| }
\hline
56 & 34 & 8 & 57 & 18 & 47 & 9 & 31 \\ 
\hline
33 & 20 & 54 & 48 & 7 & 29 & 59 & 10 \\ 
\hline
26 & 43 & 13 & 23 & 64 & 38 & 4 & 49 \\ 
\hline
19 & 5 & 35 & 30 & 53 & 12 & 46 & 60 \\ 
\hline
15 & 25 & 63 & 2 & 41 & 24 & 50 & 40 \\ 
\hline
6 & 55 & 17 & 11 & 36 & 58 & 32 & 45 \\ 
\hline
61 & 16 & 42 & 52 & 27 & 1 & 39 & 22 \\ 
\hline
44 & 62 & 28 & 37 & 14 & 51 & 21 & 3 \\ 
\hline
\end{array}$
\end{center}

Nasledovná veta dokazuje, že bimagických štvorcov je nekonečne veľa \cite{bimagic}:

\begin{theorem} (Chen, Li, 2004) Nech $m,n$ sú kladné celé čísla s rovnakou paritou, pričom $m,n \notin \{2,3,6\}$. Potom existuje superbimagický štvorec veľkosti $mn \times mn$.
\end{theorem}

Bimagické štvorce sú evidentne uzavreté na nenulový násobok. Majú však ďalšiu zaujímavú vlastnosť: sú uzavreté aj na konštantný posun. Z toho vyplýva, že vieme definovať normálne formy bimagických útvarov, ako napríklad:
\begin{itemize}
\item útvar, ktorého najmenší prvok je 1
\item útvar, ktorého magický súčet je 100
\item útvar, ktorého bimagický súčet je päťnásobkom nejakého jeho prvku
\end{itemize}

Keď predpokladáme, že bimagický štvorec je v nejakej normálnej forme, prehľadávanie sa zjednoduší. \\

\section{Multiplikatívne magické útvary}
\begin{definition} Útvar je multiplikatívny magický ak má magickú aj multiplikatívnu vlastnosť.
\end{definition}

Je zrejmé, že multiplikatívny magický štvorec veľkosti $2 \times 2$ neexistuje. Lee Morgenstern dokázal silnejšie tvrdenie \cite{multimagie}.

\begin{theorem} (Morgenstern, 2007) Neexistuje multiplikatívny magický štvorec veľkosti $3 \times 3$ ani $4 \times 4$.
\end{theorem}

Morgenstern okrem toho skúmal multiplikatívne magické štvorce veľkosti $5 \times 5$ a $6 \times 6$. V roku 2007 našiel nasledovný štvorec s jediným súčtom, ktorý nie je multiplikatívny magický \cite{multimagie}:

\begin{center}
$\begin{array}{ |c|c|c|c|c| }
\hline
105 & 182 & 40 & 198 & 45 \\ 
\hline
78 & 216 & 66 & 175 & 35  \\ 
\hline
220 & 42 & 65 & 63 & 180 \\ 
\hline
140 & 55 & 189 & 30 & 156 \\ 
\hline
27 & 75 & 210 & 104 & 154 \\ 
\hline
\end{array}$
\end{center}

Podarilo sa mu nájsť aj tento semimultiplikatívny magický štvorec (k vyvráteniu hypotézy mu chýbajú iba multiplikatívne diagonály) \cite{multimagie}:

\begin{center}
$\begin{array}{ |c|c|c|c|c|c| }
\hline
27 & 25 & 156 & 48 & 84 & 20 \\ 
\hline
75 & 144 & 18 & 56 & 52 & 15 \\ 
\hline
24 & 12 & 45 & 117 & 50 & 112 \\ 
\hline
16 & 65 & 21 & 30 & 108 & 120 \\ 
\hline
140 & 72 & 40 & 9 & 60 & 39 \\ 
\hline
78 & 42 & 80 & 100 & 6 & 54 \\
\hline
\end{array}$
\end{center}

V roku 2016 našiel Sébastien Miquel multiplikatívny magický štvorec veľkosti $7 \times 7$  \cite{multimagie}:

\begin{center}
$\begin{array}{ |c|c|c|c|c|c|c| } 
\hline
126 & 66 & 50 & 90 & 48 & 1 & 84 \\ 
\hline
20 & 70 & 16 & 54 & 189 & 110 & 6 \\ 
\hline
100 & 2 & 22 & 98 & 36 & 72 & 135 \\
\hline
96 & 60 & 81 & 4 & 10 & 49 & 165 \\
\hline
3 & 63 & 30 & 176 & 120 & 45 & 28 \\
\hline
99 & 180 & 14 & 25 & 7 & 108 & 32 \\
\hline
21 & 24 & 252 & 18 & 55 & 80 & 15 \\
\hline
\end{array}$
\end{center}

Multiplikatívne štvorce je možné nájsť napríklad vzorkovaním. Ale existencia multiplikatívneho magického štvorca veľkosti $5 \times 5$ alebo $6 \times 6$ je naďalej otvoreným problémom.











