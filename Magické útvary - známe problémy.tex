\documentclass[12pt]{article}

\usepackage[utf8]{inputenc}
\usepackage[T1]{fontenc}
\usepackage[slovak]{babel}

\usepackage{amsfonts} 
% Use the graphicx package for including images
\usepackage{graphicx}

% Use the hyperref package for clickable links within document
% and to webpages
\usepackage{hyperref}

\usepackage{listings}
\lstset{language=Python}

%Use this package for theorems and proofs
\usepackage{amsthm}
\newtheorem{theorem}{Theorem}

\usepackage{mathtools}
\DeclarePairedDelimiter\ceil{\lceil}{\rceil}
\DeclarePairedDelimiter\floor{\lfloor}{\rfloor}


% Start the text
\begin{document}
\textbf{Hypotéza 1}: Existuje jediný magický štvorec veľkosti $3 \times 3$ (spolu s jeho násobkami, rotáciami a symetriami), ktorého aspoň $7$ prvkov sú druhé mocniny prirodzených čísel. \\
$373^2, 289^2, 565^2$ \\
$360721, 425^2, 23^2$ \\
$205^2, 527^2, 222121$ \\\\

\textbf{Veta 1.1}: Nech $e$ je prostredný prvok magického štvorca veľkosti $3 \times 3$. Potom je jeho magický súčet rovný $3e$. \\

\textbf{Dôkaz}: Nech $s$ je magický súčet. Označme $a, b, .... , i$ prvky štvorca zľava doprava po jednotlivých riadkoch (čiže $e$ je prostredný z nich). Potom platí $3s = (a + e + i) + (b + e + h) + (c + e + g) = (a + b + c) + (g + h + i) + 3e = 2s + 3e$, z čoho vyplýva, že $s = 3e$. \\\\

\textbf{Dôsledok 1.1}: Nech $e$ je prostredný prvok magického štvorca veľkosti $3 \times 3$ a $x,y$ sú jeho ľubovoľné dva protiľahlé prvky. Potom $x + y = 2e$. \\\\

\textbf{Veta 1.2}: Nech $u_1, v_1, u_2, v_2$ sú navzájom rôzne kladné celé čísla. Definujme hodnoty $p,q,r,s,t$ nasledovne: \\
$p = (u_1^2 + v_1^2)(u_2^2 + 2u_2 v_2 - v_2^2)$ \\
$q = (u_1^2 + 2u_1 v_1 - v_1^2)(u_2^2 + v_2^2)$ \\
$r = (- u_1^2 + 2u_1 v_1 + v_1^2)(u_2^2 + v_2^2)$ \\
$s = (u_1^2 + v_1^2)(-u_2^2 + 2u_2 v_2 + v_2^2)$ \\
$t = (u_1^2 + v_1^2)(u_2^2 + v_2^2)$ \\

Potom: \\
$p^2, 3t^2 - p^2 - q^2, q^2$ \\
$3t^2 - p^2 - r^2, t^2, 3t^2 - q^2 - s^2$ \\
$r^2, 3t^2 - r^2 - s^2, s^2$ \\
.\\
$\frac{(r^2 + s^2)}{2}, p^2, \frac{(q^2 + s^2)}{2}$ \\
$q^2, t^2, r^2$ \\
$\frac{(p^2 + r^2)}{2}, s^2, \frac{(p^2 + q^2)}{2}$ \\
.\\
$p^2, q^2, 3t^2 - p^2 - q^2$ \\
$r^2 + s^2 - p^2, t^2, p^2 + q^2 - s^2$ \\
$3t^2 - r^2 - s^2, r^2, s^2$ \\

sú magické štvorce, ktorých aspoň $5$ prvkov sú druhé mocniny prirodzených čísel. \\

\textbf{Veta 1.3}: Nech $x$ je kladné celé číslo. Potom sú tieto štvorce: \\
$(2x^5 + 4x^3 - 7x)^2, (x^2 - 2)(8x^2 - 1)(x^6 - 6x^4 - 2), (5x^4 - 2x^2 + 2)^2$ \\
$(x^4 + 8x^2 - 2)^2, (2x^5 - 2x^3 + 5x)^2, (x^2 - 2)(8x^8 - x^6 + 30x^4 - 40x^2 + 2)$ \\
$(x^2 - 2)(8x^8 - 25x^6 + 18x^4 - 28x^2 + 2), (7x^4 - 4x^2 - 2)^2, (2x^5 - 8x^3 - x)^2$ \\
.\\
$(5x^4 - 2x^2 + 2)^2, (2x^5 + 4x^3 - 7x)^2, \frac{4x^{10} - 31x^8 + 76x^6 + 76x^4 - 31x^2 + 4}{2}$ \\
$(2x^5 - 8x^3 - x)^2, \frac{4x^{10} + 17x^8 + 4x^6 + 4x^4 + 17x^2 + 4}{2}, (7x^4 - 4x^2 - 2)^2$ \\
$\frac{4x^{10} + 65x^8 - 68x^6 - 68x^4 + 65x^2 + 4}{2}, (x^4 + 8x^2 - 2)^2, (2x^5 - 2x^3 + 5x)^2$ \\

magickými štvorcami veľkosti $3 \times 3$, ktorých aspoň $6$ prvkov sú druhé mocniny prirodzených čísel. \\\\

\textbf{Hypotéza 2}: Neexistuje bimagický štvorec veľkosti $5 \times 5$. \\

\textbf{Veta 2.1}: Neexistuje bimagický štvorec veľkosti $3 \times 3$. \\

\textbf{Dôkaz}: Sporom. Nech $a,b$ sú prvky v prvom riadku a prvých dvoch stĺpcoch. Nech $c,d$ sú prvky v poslednom stĺpci a posledných dvoch riadkoch. Nech $x$ je prvok v prvom riadku a poslednom stĺpci. Potom musia platiť vzťahy $a + b + x = x + c + d$ aj $a^2 + b^2 + x^2 = x^2 + c^2 + d^2$. Tým dostaneme sústavu z duplikačnej lemy, z čoho vyplýva, že $c = a$ alebo $c = b$, čo je spor. \\\\

\textbf{Veta 2.2}: Neexistuje bimagický štvorec veľkosti $4 \times 4$. \\\\

\textbf{Dôkaz}: Sporom. Nech $a, b, ... , o, p$ sú prvky zľava doprava v jednotlivých riadkoch štvorca. Keďže štvorec je magický, musia platiť nasledovné vzťahy: \\
$a + b + c + d = m + n + o + p$ \\
$a + f + k + p = b + f + j + n$ \\
$d + g + j + m = c + g + k + o$ \\

Ich sčítaním dostaneme $a + d = n + o$. Keďže štvorec je zároveň aj multiplikatívny, musia platiť nasledovné vzťahy: \\
$a^2 + b^2 + c^2 + d^2 = m^2 + n^2 + o^2 + p^2$ \\
$a^2 + f^2 + k^2 + p^2 = b^2 + f^2 + j^2 + n^2$ \\
$d^2 + g^2 + j^2 + m^2 = c^2 + g^2 + k^2 + o^2$ \\

Ich sčítaním dostaneme $a^2 + d^2 = n^2 + o^2$. Tým dostaneme sústavu z duplikačnej lemy, z čoho vyplýva, že $n = a$ alebo $n = d$, čo je spor. \\\\

\textbf{Veta 2.3}: Neexistuje bimagický štvorec veľkosti $5 \times 5$, ktorého prvky sú čísla od $1$ do $1500$. \\\\

\textbf{Veta 2.4}: Nech $A$ je semibimagický štvorec veľkosti $5 \times 5$. Potom existuje číslo $x \in \mathbb{N}$, pre ktoré vieme zostrojiť semibimagický štvorec $B$ rovnakej veľkosti, pričom platí: \\
(i) v prvom riadku $B$ sú v poradí prvky $x, a+b-c, a-b+c, -a+b+c, -a-b-c$, pričom $a,b,c \in \mathbb{N}$ \\   
(ii) v prvom stĺpci $B$ sú v poradí prvky $x, d+e-f, d-e+f, -d+e+f, -d-e-f$, pričom $d,e,f \in \mathbb{N}$ \\

\textbf{Dôkaz}: Uvažujme magický štvorec veľkosti $5 \times 5$, ktorého súčet prvých $4$ prvkov v prvom riadku je rovný $0$. Ak sú prvé $3$ prvky $A, B, C$, tak posledný musí byť $-A-B-C$. Ich bimagický súčet je potom $A^2 + B^2 + C^2 + (-A-B-C)^2 = (A+B)^2 + (A+C)^2 + (B+C)^2$. Nech $a = A+B, b = A+C, c = B+C$. Potom $A = \frac{a+b-c}{2}, B = \frac{a-b+c}{2}, C = \frac{a+b-c}{2}$. Rovnako odvodíme, že ak sú v poslednom stĺpci prvé $3$ prvky $D, E, F$, tak $D = \frac{d+e-f}{2}, E = \frac{d-e+f}{2}, F = \frac{d+e-f}{2}$ pre vhodné $d,e,f$. Všetky prvky $A, B, C, D, E, F$ vynásobíme $2$ a môžeme ich v danom riadku alebo stĺpci ľubovoľne premiestňovať (keďže sme v bimagickom štvorci). \\\\

\textbf{Veta 2.5}: Nech $A$ je bimagický štvorec veľkosti $5 \times 5$. Potom existuje číslo $x \in \mathbb{N}$, pre ktoré vieme zostrojiť semibimagický štvorec $B$ rovnakej veľkosti, pričom platí, že v ľavom dolnom rohu $B$ je prvok $x$ a v pravom dolnom rohu prvok $-x$. \\\\

\textbf{Hypotéza 3}: Neexistuje multiplikatívny magický štvorec veľkosti $5 \times 5$ alebo $6 \times 6$. \\\\

\textbf{Veta 3.1}: Neexistuje multiplikatívny magický štvorec veľkosti $3 \times 3$. \\

\textbf{Dôkaz}: Sporom. Nech $a,b$ sú prvky v prvom riadku a prvých dvoch stĺpcoch. Nech $c,d$ sú prvky v poslednom stĺpci a posledných dvoch riadkoch. Nech $x$ je prvok v prvom riadku a poslednom stĺpci. Potom musia platiť vzťahy $a + b + x = x + c + d$ aj $abx = xcd$. Tým dostaneme sústavu z duplikačnej lemy, z čoho vyplýva, že $c = a$ alebo $c = b$, čo je spor.  \\\\

\textbf{Veta 3.2}: Neexistuje multiplikatívny magický štvorec veľkosti $4 \times 4$. \\

\textbf{Dôkaz}: Sporom. Nech $a, b, ... , o, p$ sú prvky zľava doprava v jednotlivých riadkoch štvorca. Keďže štvorec je magický, musia platiť nasledovné vzťahy: \\
$a + b + c + d = m + n + o + p$ \\
$a + f + k + p = b + f + j + n$ \\
$d + g + j + m = c + g + k + o$ \\

Ich sčítaním dostaneme $a + d = n + o$. Keďže štvorec je zároveň aj multiplikatívny, musia platiť nasledovné vzťahy: \\
$abcd = mnop$ \\
$afkp = bfjn$ \\
$dgjm = cgko$ \\

Ich vynásobením dostaneme $ad = no$. Tým dostaneme sústavu z duplikačnej lemy, z čoho vyplýva, že $n = a$ alebo $n = d$, čo je spor. \\\\

\textbf{Veta 3.3}: Nech $A$ je multiplikatívny štvorec veľkosti $5 \times 5$ alebo $6 \times 6$, $V$ je ľubovoľná jeho vzorka a $n$ je kladné celé číslo. Nech $B$ je štvorec, ktorý vznikne prenásobením vzorky $V$ číslom $n$. Potom $B$ je multiplikatívny štvorec. \\


\end{document}