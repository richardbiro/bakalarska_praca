\documentclass[12pt]{article}

\usepackage[utf8]{inputenc}
\usepackage[T1]{fontenc}
\usepackage[slovak]{babel}

\usepackage{amsfonts} 
% Use the graphicx package for including images
\usepackage{graphicx}

% Use the hyperref package for clickable links within document
% and to webpages
\usepackage{hyperref}

\usepackage{listings}
\lstset{language=Python}

%Use this package for theorems and proofs
\usepackage{amsthm}
\newtheorem{theorem}{Theorem}

\usepackage{mathtools}
\DeclarePairedDelimiter\ceil{\lceil}{\rceil}
\DeclarePairedDelimiter\floor{\lfloor}{\rfloor}


% Start the text
\begin{document}

\textbf{Algoritmy pre magické štvorce} \\

\textbf{Algoritmus 1}: Na vstupe dostaneme navzájom rôzne kladné celé čísla $u_1, v_1, u_2, v_2 \in \mathbb{N}$. Výstupom je magický štvorec veľkosti $3 \times 3$, ktorého aspoň $7$ prvkov sú druhé mocniny prirodzených čísel. Algoritmus využije tri parametrické vzorce z vety 1.2, ktoré generujú vyhovujúce magické štvorce. \\

\textbf{Pseudokód}: \\
def otestuj($u_1, v_1, u_2, v_2$): \\
$p = (u_1^2 + v_1^2)(u_2^2 + 2u_2 v_2 - v_2^2)$ \\
$q = (u_1^2 + 2u_1 v_1 - v_1^2)(u_2^2 + v_2^2)$ \\
$r = (- u_1^2 + 2u_1 v_1 + v_1^2)(u_2^2 + v_2^2)$ \\
$s = (u_1^2 + v_1^2)(-u_2^2 + 2u_2 v_2 + v_2^2)$ \\
$t = (u_1^2 + v_1^2)(u_2^2 + v_2^2)$ \\
ak sú aspoň dva z $3t^2 - p^2 - q^2, 3t^2 - p^2 - r^2, 3t^2 - q^2 - s^2, 3t^2 - r^2 - s^2$ štvorce RETURN prvý štvorec \\
ak sú aspoň dva z $3t^2 - p^2 - q^2, 3t^2 - p^2 - r^2, 3t^2 - q^2 - s^2, 3t^2 - r^2 - s^2$ štvorce RETURN druhý štvorec \\
ak sú aspoň dva z $3t^2 - p^2 - q^2, r^2 + s^2 - p^2, p^2 + q^2 - s^2, 3t^2 - r^2 - s^2$ štvorce RETURN tretí štvorec \\

\textbf{Výsledky}: Pre $u_1, v_1, u_2, v_2 < 1000$ nedokážu parametrické vzorce vygenerovať magický štvorec veľkosti $3 \times 3$, ktorého aspoň $7$ prvkov sú druhé mocniny prirodzených čísel. \\\\

\textbf{Algoritmus 2}: Na vstupe dostaneme kladné celé číslo $x \in \mathbb{N}$. Výstupom je magický štvorec veľkosti $3 \times 3$, ktorého aspoň $7$ prvkov sú druhé mocniny prirodzených čísel. Algoritmus využije dva parametrické vzorce z vety 1.3, ktoré generujú vyhovujúce magické štvorce. \\

\textbf{Pseudokód}: \\
def otestuj(x): \\
ak je $(x^2 - 2)(8x^2 - 1)(x^6 - 6x^4 - 2)$ druhou mocninou RETURN prvý štvorec \\
ak je $(x^2 - 2)(8x^8 - x^6 + 30x^4 - 40x^2 + 2)$ druhou mocninou RETURN prvý štvorec \\
ak je $(x^2 - 2)(8x^8 - 25x^6 + 18x^4 - 28x^2 + 2)$ druhou mocninou RETURN prvý štvorec \\
ak je $\frac{4x^{10} - 31x^8 + 76x^6 + 76x^4 - 31x^2 + 4}{2}$ druhou mocninou RETURN druhý štvorec \\
ak je $\frac{4x^{10} + 17x^8 + 4x^6 + 4x^4 + 17x^2 + 4}{2}$ druhou mocninou RETURN druhý štvorec \\
ak je $\frac{4x^{10} + 65x^8 - 68x^6 - 68x^4 + 65x^2 + 4}{2}$ druhou mocninou RETURN druhý štvorec \\

\textbf{Výsledky}: Pre $x < 10^8$ nedokážu parametrické vzorce vygenerovať magický štvorec veľkosti $3 \times 3$, ktorého aspoň $7$ prvkov sú druhé mocniny prirodzených čísel. \\\\

\textbf{Algoritmy pre vrcholovo bimagické grafy} \\
 
\textbf{Algoritmus 1}: Na vstupe dostaneme ľubovoľný súvislý graf. Výstupom je odpoveď, či má graf šancu byť vrcholovo bimagickým. Pre každú dvojicu jeho vrcholov overíme, či spĺňa podmienku z vety 1.4. Ak existuje dvojica vrcholov, pre ktorú graf nevyhovuje niektorej z podmienok (i) - (iii), tak môžeme s istotou povedať, že nie je vrcholovo bimagický. \\

\textbf{Pseudokód}: \\
def otestuj(G): \\
pre všetky dvojice vrcholov $v_1, v_2$ grafu $G$ \\
x = |\{susedia[v1]\} - \{susedia[v2]\}| \\
y = |\{susedia[v2]\} - \{susedia[v1]\}| \\
ak $(xy = 0$ a $x+y > 0)$ RETURN graf $G$ nie je vrcholovo bimagický \\
ak $x = 1$ alebo $y = 1$ RETURN graf $G$ nie je vrcholovo bimagický \\
ak $x = y = 2$ RETURN graf $G$ nie je vrcholovo bimagický \\

\textbf{Výsledky}: jediné súvislé grafy s menej ako $10$ vrcholmi, ktoré spĺňajú všetky podmienky (a teda môžu byť vrcholovo bimagickými), sú: \\
$K_{2,3}$ \\
$K_{2,4}, K_{3,3}$ \\
$K_{2,5}, K_{3,4}$ \\
$K_{2,6}, K_{3,5}, K_{4,4}, K_{2,3,3}$ \\
$K_{2,7}, K_{3,6}, K_{4,5}, K_{2,3,4}, K_{3,3,3}$ \\

Vieme, že $K_{i,j}$ je vrcholovo bimagický pre $i,j \geq 2, (i,j) \neq (2,2)$. Môžeme sa ľahko presvedčiť, že aj zvyšné grafy majú vrcholové bimagické ohodnotenie: \\
$K_{2,3,3} \rightarrow 11, 13 ~|~ 1, 8, 15 ~|~ 3, 5, 16$ \\
$K_{2,3,4} \rightarrow 11, 19 ~|~ 1, 9, 20 ~|~ 1, 2, 6, 21$ \\
$K_{3,3,3} \rightarrow 1, 12, 14 ~|~ 2, 9, 16 ~|~ 4, 6, 17$ \\\\

\textbf{Algoritmus 2}: Na vstupe dostaneme kompletný bipartitný graf $K_{i,j}$. Výstupom má byť vrcholové bimagické ohodnotenie tohto grafu. Algoritmus bude replikovať indukčný dôkaz vety 1.5. \\

\textbf{Pseudokód}: \\
def ohodnot(i,j): \\
ak $i > j$ RETURN ohodnot(j,i) \\
ak $i \leq 1$ alebo $i = j = 2$ RETURN nie je možné ohodnotiť graf $K_{i,j}$ \\
ak $i = 2$ RETURN $(\frac{j(j-1)}{2} + 1, \frac{j(j-1)(3j^2-7j+14)}{24}), (1, ... , j-1, \frac{j(j-1)(3j^2-7j+14)}{24} + 1)$ \\
ak $i = 3$ RETURN $(1, \frac{j(j+1)}{2} - 1, \frac{j(j+1)(3j^2-j-14)}{24} + 1), (2, ... , j, \frac{j(j+1)(3j^2-j-14)}{24} + 2)$ \\
ak $(i, j) = (4,4)$ RETURN $(1, 4, 6, 7), (2, 3, 5, 8)$ \\
ak $(i, j) = (4,5)$ RETURN $(2, 12, 13, 15), (1, 4, 8, 10, 19)$ \\
H = ohodnot(i - 2, j - 3); m = max(H) + 1 \\
na ľavú stranu H pridaj $4m, 5m$, na pravú stranu H pridaj $m, 2m, 6m$ \\
RETURN H \\\\

\textbf{Algoritmus 3}: Na vstupe dostaneme číslo $n \in \mathbb{N}$. Výstupom algoritmu má byť vrcholové superbimagické ohodnotenie kompletného bipartitného grafu s $n$ vrcholmi. Algoritmus bude replikovať indukčný dôkaz vety 1.9. \\

\textbf{Pseudokód}: \\
def ohodnotSuper(n): \\
ak $n < 7$ RETURN nie je možné ohodnotiť \\
ak dáva $n$ po delení $4$ zvyšok $1$ alebo $2$ RETURN nie je možné ohodnotiť \\
ak $n = 7$ RETURN (1, 2, 4, 7), (3, 5, 6) \\
ak $n = 8$ RETURN (1, 4, 6, 7), (2, 3, 5, 8) \\
ak $n = 11$ RETURN (1, 3, 4, 5, 9, 11), (2, 6, 7, 8, 10) \\
ak $n = 12$ RETURN (1, 3, 7, 8, 9, 11), (2, 4, 5, 6, 10, 12) \\
H = ohodnotSuper(n - 8) \\
pre $x$ od $1$ po $8$ vrátane \\
ak je $x$ z množiny $1, 4, 6, 7$ pridaj $(n-8)+x$ na ľavú stranu H \\
ak je $x$ z množiny $2, 3, 5, 8$ pridaj $(n-8)+x$ na pravú stranu H \\
po skončení RETURN H \\\\

\textbf{Algoritmy pre vrcholovo multiplikatívne magické grafy} \\

\textbf{Algoritmus 1}: Na vstupe dostaneme kompletný bipartitný graf $K_{i,j}$. Výstupom má byť vrcholové multiplikatívne magické ohodnotenie tohto grafu. Algoritmus bude replikovať indukčný dôkaz vety 1.5. \\

\textbf{Pseudokód}: \\
def ohodnot(i,j): \\
ak $i > j$ RETURN ohodnot(j,i) \\
ak $i \leq 1$ alebo $i = j = 2$ RETURN nie je možné ohodnotiť graf $K_{i,j}$ \\
ak $(i, j) = (2,3)$ RETURN $(5, 12), (1, 6, 10)$ \\
ak $(i, j) = (2,4)$ RETURN $(9, 16), (1, 2, 4, 18)$ \\
ak $i = 2$ RETURN $((j-1)! + 1, (j-1)! ((j-1)! + 1 - \frac{j(j-1)}{2}), (1, ... , j-1, ((j-1)! + 1) ((j-1)! + 1 - \frac{j(j-1)}{2}))$ \\
ak $i = 3$ RETURN $(1, j! + 1, j! (j! + 3 - \frac{j(j+1)}{2}), (2, ... , j, (j! + 1) (j! + 3 - \frac{j(j+1)}{2}))$ \\
ak $(i, j) = (4,4)$ RETURN $(1, 5, 6, 12), (2, 3, 4, 15)$ \\
ak $(i, j) = (4,5)$ RETURN $(2, 10, 20, 27), (1, 3, 6, 24, 25)$ \\
H = ohodnot(i - 2, j - 3); x = max(H) + 1; y = max(H) + 2 \\
na ľavú stranu H pridaj $2xy, 2xy - x - y$ \\
na pravú stranu H pridaj $2(2xy - x - y), x, y$ \\
RETURN H \\\\


\textbf{Algoritmy pre bimagické obdĺžniky} \\

\textbf{Algoritmus 1}: Na vstupe dostaneme číslo $n,h \in \mathbb{N}, n \geq 4$. Výstupom má byť bimagický obdĺžnik veľkosti $3 \times n$, ktorého prvky sú kladné celé čísla neprevyšujúce $h$. Náš algoritmus predpokladá, že najmenší prvok obdĺžnika je $1$. Pre každú trojicu rôznych celých čísel $a,b,c$ väčších ako $1$ si predpočíta ich magický a bimagický súčet. Ak medzi súčtami platí istý vzťah, potom je možné nájsť celé čísla $d,e$ tak, aby mohli byť trojice $(a,b,c)$ a $(1,d,e)$ použité ako stĺpce v tom istom bimagickom obdĺžniku. Pre každú takú trojicu $(a,b,c)$ si algoritmus uloží hodnoty $(1,d,e)$ ako kľúč do asociatívneho poľa. Potom toto pole prejde a v každom kľúči vyberie $n-1$ rôznych zapamätaných trojíc (ku ktorým pridá trojicu v kľúči). \\

\textbf{Pseudokód}: \\
def ohodnot(n,h): \\
pre $a$ od $2$ po $h$ vrátane \\
pre $b$ od $a+1$ po $h$ vrátane \\
pre $c$ od $b+1$ po $h$ vrátane \\
$s = a+b+c; t = a^2+b^2+c^2$ \\
ak je $2t - (s-1)^2 - 2$ druhou mocninou celého čísla a má inú paritu ako $s$ \\
pridám do asociatívneho poľa $D$ trojicu $(a,b,c)$ pre kľúč $(1, \frac{s-1 + sqrt(2t - (s-1)^2 - 2)}{2}, \frac{s-1 - sqrt(2t - (s-1)^2 - 2)}{2})$ \\
po skončení pre každý kľúč $k$ v $D$ \\
pre každú $(n-1)$-prvkovú podmnožinu trojíc v D[k] \\
ak sú všetky vybraté čísla navzájom rôzne \\
prejdi všetky permutácie každej trojice \\
kľúč $k$ a $n-1$ trojíc v danom poradí ulož vedľa seba do stĺpcov \\
ak má vzniknutý obdĺžnik bimagické riadky, vypíš ho \\

\textbf{Výsledky}: Neexistuje bimagický obdĺžnik veľkosti $3 \times n$, ktorého všetky prvky neprevyšujú $400$. \\\\

\textbf{Algoritmus 2}: Na vstupe dostaneme čísla $n,s \in \mathbb{N}, n \geq 4$. Výstupom má byť bimagický obdĺžnik veľkosti $3 \times n$, ktorého prvky sú kladné celé čísla, pričom ich súčet v každom stĺpci je $s$. Náš algoritmus predpokladá, že najmenší prvok obdĺžnika je $1$. Pre každú trojicu rôznych celých čísel $a,b,c$ ($1 < a < b < c, a+b+c = s$) si predpočíta ich bimagický súčet. Ak platí istý vzťah, potom je možné nájsť celé čísla $d,e$ tak, aby mohli byť trojice $(a,b,c)$ a $(1,d,e)$ použité ako stĺpce v tom istom bimagickom obdĺžniku. Pre každú takú trojicu $(a,b,c)$ si algoritmus uloží hodnoty $(1,d,e)$ ako kľúč do asociatívneho poľa. Potom toto pole prejde a v každom kľúči vyberie $n-1$ rôznych zapamätaných trojíc (ku ktorým pridá trojicu v kľúči). \\

\textbf{Pseudokód}: \\
def ohodnot(n,s): \\
pre $a$ od $2$ po $\frac{s}{3}$ vrátane \\
pre $b$ od $a+1$ po $\frac{s-a}{2}$ vrátane \\
$c = s-a-b; t = a^2+b^2+c^2$ \\
pokračujem rovnako ako predchádzajúci algoritmus \\

\textbf{Výsledky}: Neexistuje bimagický obdĺžnik veľkosti $3 \times n$, ktorého súčet prvkov v riadku je menší ako $384$. Podarilo sa nájsť niekoľko magických obdĺžnikov veľkosti $3 \times 6$, $3 \times 8$ a $3 \times 10$ s bimagickými stĺpcami a jediným nebimagickým riadkom. Najmenší z nich má súčet v stĺpci rovný $144$: \\
1, 3, 88, 8, 93, 95 \\
63, 56, 51, 91, 11, 16 \\
80, 85, 5, 45, 40, 33 \\\\

\textbf{Algoritmus 3}: Na vstupe dostaneme číslo $n \in \mathbb{N}$. Výstupom má byť bimagický obdĺžnik veľkosti $3 \times n$, ktorého prvky sú celé (potenciálne záporné) čísla v absolútnej hodnote neprevyšujúce $h$. Náš algoritmus predpokladá, že bimagický obdĺžnik má v každom riadku aj stĺpci nulový súčet. Trojica prvkov v každom stĺpci je preto v tvare $a, b, -a-b$. Pre každú dvojicu celých čísel $a,b$ si algoritmus uloží hodnotu výrazu $a^2 + b^2 + (-a-b)^2$ ako kľúč do asociatívneho poľa. Potom toto pole prejde a v každom kľúči vyberie $n$ rôznych zapamätaných dvojíc. \\

\textbf{Pseudokód}: \\
def ohodnot(n,h): \\
pre $a$ od $0$ po $h$ vrátane \\
pre $b$ od $-a+1$ po $a-1$ vrátane \\
$t = a^2 + b^2 + (-a-b)^2$ \\
pridám do asociatívneho poľa $D$ dvojicu $(a,b)$ pre kľúč $t$ \\
po skončení pre každý kľúč $k$ v $D$ \\
pre každú $n$-prvkovú podmnožinu dvojíc v D[k] \\
z každej dvojice $(a,b)$ zrekonštruuj trojicu $(a,b,-a-b)$ \\
ak sú všetky vybraté čísla navzájom rôzne \\
prejdi všetky permutácie každej trojice (ber do úvahy aj opačné znamienka) \\
$n$ trojíc v danom poradí ulož vedľa seba do stĺpcov \\
ak má vzniknutý obdĺžnik bimagické riadky, vypíš ho \\\\

\textbf{Algoritmy pre multiplikatívne magické obdĺžniky} \\

\textbf{Algoritmus 1}: Na vstupe dostaneme čísla $n,h \in \mathbb{N}, n \geq 4$. Výstupom má byť multiplikatívny magický obdĺžnik veľkosti $3 \times n$, ktorého prvky sú kladné celé čísla neprevyšujúce $h$. Vieme, že obdĺžnik nemôže obsahovať prvočíslo $p$, pre ktoré platí $pn > h$ (inak by sme mali nanajvýš $n-1$ násobkov $p$, ktoré by sme museli vedieť rozdeliť do $n$ stĺpcov, čo je spor). Náš algoritmus si pre každú trojicu vyhovujúcich rôznych kladných čísel predpočíta ich súčet a súčin a obe hodnoty si uloží ako kľúč do asociatívneho poľa. Potom toto pole prejde a v každom kľúči vyberie $n$ rôznych zapamätaných trojíc. \\

\textbf{Pseudokód}: \\
def ohodnot(n,h): \\
vyhovuju = $ \{x ~|~ x \in \{1, ... , h\}, x$ nie je prvočíslo alebo $xn \leq h\}$ \\
pre všetky trojice rôznych vyhovujúcich čísel $a,b,c$ \\
$s = a+b+c; p = abc$ \\
pridám do asociatívneho poľa $D$ trojicu $(a,b,c)$ pre kľúč $(s,p)$ \\
po skončení pre každý kľúč $k$ v $D$ \\
pre každú $n$-prvkovú podmnožinu trojíc v D[k] \\
ak sú všetky vybraté čísla navzájom rôzne \\
prejdi všetky permutácie pre druhú, tretiu, ... , $n$-tú trojicu \\
$n$ trojíc v danom poradí ulož vedľa seba do stĺpcov \\
ak má vzniknutý obdĺžnik multiplikatívne magické riadky, vypíš ho \\

\textbf{Algoritmus 2}: Na vstupe dostaneme čísla $n,s \in \mathbb{N}, n \geq 4$. Výstupom má byť multiplikatívny magický obdĺžnik veľkosti $3 \times n$, ktorého prvky sú kladné celé čísla, pričom ich súčet v každom stĺpci je $s$. Vieme, že obdĺžnik nemôže obsahovať prvočíslo $p$, pre ktoré platí $pn > s$. Náš algoritmus si pre každú trojicu vyhovujúcich rôznych kladných čísel si ich súčin uloží ako kľúč do asociatívneho poľa. Potom toto pole prejde a v každom kľúči vyberie $n$ rôznych zapamätaných trojíc. \\

\textbf{Pseudokód}: \\
def ohodnot(n,s): \\
vyhovuju = $ \{x ~|~ x \in \{1, ... , s\}, x$ nie je prvočíslo alebo $xn \leq s\}$ \\
pre všetky dvojice rôznych vyhovujúcich čísel $a,b$ \\
ak $c = s-a-b$ je vyhovujúce \\
$p = abc$ \\
pridám do asociatívneho poľa $D$ trojicu $(a,b,c)$ pre kľúč $p$ \\
po skončení pokračujem rovnako ako predchádzajúci algoritmus \\

\textbf{Výsledky}: Neexistuje multiplikatívny magický obdĺžnik veľkosti $3 \times n$, ktorého súčet prvkov v riadku je menší ako $4000$. Podarilo sa nájsť niekoľko multiplikatívnych obdĺžnikov veľkosti $3 \times 6$ a $3 \times 9$ s magickými stĺpcami. Najmenší z nich má súčet v stĺpci rovný $485$: \\
14, 294, 16, 385, 60, 396 \\
231, 15, 154, 72, 392, 40 \\
240, 176, 315, 28, 33, 49 \\

\textbf{Poznámka}: Algoritmy pre multiplikatívne magické obdĺžniky sa dajú obmedziť tak, aby dovoľovali iba konečný počet prvočísel v prvočíselnom rozklade.



\end{document}